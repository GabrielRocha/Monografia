\begin{center}
\textbf{ABSTRACT}
\end{center}

\singlespacing

\noindent This work suggests a proposal to implant a files and printers sharing server and a domain controller in an educational institution, with mission to facilitate the sharing of available network resources, making safer and reliable the control of user’s access to these resources. More specifically, this work implants the Samba software, in versions 3 and 4, as a free alternative to 
proprietary solutions such as Microsoft Active Directory, serving the resources above to both Windows and Linux clients. Will also be presented basic concepts for the understanding of the tools used in addition to step-by-step instructions and scripts needed to carry out the implementation of the entire structure proposed in this work. Will be addressed in this work, a case study of the use of the Samba software in Instituto Federal fluminense in Bom Jesus do Itabapoana, showing the tool being configured in a real case. By the end we will present the findings on the technology and future work.\\

\noindent KEYWORDS:  Samba 3, Samba 4, Primary Domain Controller, File Server, Print Server, Open Source