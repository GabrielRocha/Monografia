\chapter{ESTUDO DE CASO}

Esta proposta de implementação foi motivada através de um cenário de uma instituição de ensino que necessitava de uma otimização na segurança e compartilhamento de seus recursos de TI. Para melhor gerenciamento e manutenção dos arquivos compartilhados e usuários na rede, seria necessário a implantação de um servidor que centralizasse todas essas tarefas.

%Após identificada a necessidade desse novo recurso, foi iniciada uma pesquisa para encontrar um software que atendesse os requisitos. O Windows Server em todas as suas versões até hoje lançadas poderia ser a solução, mas é proprietário e o valor de uma licença da versão 2012 \textit{Datacenter} custa, atualmente, em torno de 10 mil reais \cite{SERVER}. O alto valor da licença acaba inviabilizando a utilização da mesma nas instituições  de ensino e em pequenas empresas. 
Foi iniciada uma pesquisa para encontrar um software que atendesse a todos requisitos. O Windows Server é uma solução, mas é proprietário e o valor de uma licença da versão 2012 \textit{Datacenter} custa em torno de 10 mil reais \cite{SERVER}. O alto valor da licença acaba inviabilizando a utilização em instituições de ensino e em pequenas empresas. 
Para solucionar esse problema da compra de licenças foi criada uma versão livre, o Samba 4, que faz as mesmas tarefas de um Windows Server, trabalhando com o mesmo protocolo, o LDAP. Pelo custo benefício, o Samba 4 foi utilizada neste trabalho.
%A instituição abordada neste trabalho contém 110 computadores nos setores administrativos e 90 nos laboratórios de informática. Abaixo uma pequena demonstração da estrutura da rede \ref{rede}:
A instituição contém 110 computadores nos setores administrativos e 90 nos laboratórios de informática. Abaixo uma demonstração da estrutura da rede \ref{rede}:

\begin{figure}[h!]
   	\centering
    \scalebox{1}{\includegraphics{figuras/iff}}
   	\caption{Estrutura da rede do instituto}
    \label{rede}
\end{figure}

\pagebreak
          				
Os setores são divididos conforme suas funções no organograma da instituição. Os principais são:

* Diretoria do Departamento de Administração e Finanças

* Diretoria do Departamento de Gestão de Pessoas

* Coordenação de Registros Acadêmicos

* Chefe de Gabinete

Com a proposta de implementação abordada neste trabalho, cada setor e usuário terá na rede um compartilhamento próprio, com suas permissões definidas. Um servidor foi inserido na rede com o sistema operacional Debian 6.0.5 e com as seguintes configurações:

\begin{itemize}
	\item{Processador Intel Core I7\textregistered}
	\item{4GB de memória RAM}
	\item{Um servidor com 6 Tb de HD}
	\item{Placas de vídeo, áudio e rede Onboard}
\end{itemize}

Antes da instalação do Samba 4 seus pré requisitos foram instalados e o Kerberos Heimdal com suas variáveis de ambiente.
Após a configuração dos sistemas básicos, o Samba 4 foi configurado com os seguintes parâmetros.\\

\begin{lstlisting}
# cd /usr/local/samba/

# sbin/provision --use-ntvfs --realm=instituto.ensino --domain=instituto --adminpass='Senha12' --server-role='domain controller'
\end{lstlisting}

Com o samba 4 as configurações básicas realizadas, foram feitas as modificações necessárias para que fosse utilizado o servidor de dns \textit{default} do samba 4. Foi inserido no domínio do \textit{Active Directory} todas as máquinas Windows XP, através do processo manual e as máquinas Linux, através do script smbad.sh, que se encontram na rede.

Por não ter uma ferramenta mais completa para o gerenciamento do Samba 4 pelo Linux, um computador com Windows XP foi designado para tal tarefa. Nele foram instalados o adminpack e o gerenciador de gpo do Windows. Por trabalharem com o mesmo protocolo como já foi dito anteriormente não houveram incompatibilidades na utilização das ferramentas.

Os usuários foram criados a partir da interface gráfica do adminpack no Windows, respeitando os requisitos de nome completo, ramal da sala, sala, entre outras informações que auxiliam na identificação dos usuários no AD e inseridos nos respectivos grupos dos seus setores.

Com os usuários cadastrados e inseridos em seus grupos, foram criadas as GPO`s com os scripts de inicialização e nelas foram definidos os mapeamentos automáticos dos compartilhamentos

Foram criados compartilhamentos com os nomes dos setores mais importantes da instituição afim de melhorar e garantir o melhor trabalho das pessoas no setor. Com a intenção de melhorar o controle dos recursos de armazenamento foram impostas regras de QUOTA com o EDQUOTA que consiste em um dos principais programas gerenciadores de cota de disco no linux.

\begin{itemize}
		\item {Pasta do usuário: 20Gb}
		\item {Pasta do setor: 100Gb}
\end{itemize}

A seguir é apresentada uma parte do smb.conf do Samba 4, que corresponde as seções de compartilhamento de arquivos. As seções foram inseridas com a sigla dos setores. Foi decidido vetar arquivos de vídeo e áudio para não sobrecarregar o servidor.\\

\begin{lstlisting}
[Chefia_de_Gabinete]

comment = Chefia de gabinete

path = /srv/samba/chefia

valid users = usuario1, usuario2 # Usuarios do setor

read only = no

browseable = no

veto files = *.wmv / *.avi / *.wma / *.mp? / *.flv

[DDAF] 

comment = Diretoria do Departamento de Administracao e Financas

path = /srv/samba/ddaf

valid users = usuario3, usuario4 # Usuarios do setor

read only = no

browseable = no

veto files = *.wmv / *.avi / *.wma / *.mp? / *.flv

[DDGP] 

comment = Diretoria do Departamento de Gestao de Pessoas

path = /srv/samba/ddgp

valid users = usuario5, usuario6 # Usuarios do setor

read only = no

browseable = no

veto files = *.wmv / *.avi / *.wma / *.mp? / *.flv

[CRA] 

comment = Coordenacao de Registros Academicos

path = /srv/samba/cra

valid users = usuario7, usuario8 # Usuarios do setor

read only = no

browseable = no

veto files = *.wmv / *.avi / *.wma / *.mp? / *.flv


[HOME] 

comment = Pasta dos usuarios

path = /srv/samba/%U

valid users = %U

read only = no

browseable = no

veto files = *.wmv / *.avi / *.wma / *.mp? / *.flv
\end{lstlisting}

Com as sessões criadas no samba, as pastas foram criadas no /srv e atribuídas as permissões 770 com o proprietário root e o GID do grupo criado no \textit{Active Directory} com o nome do setor que foi designada a pasta:\\

\begin{lstlisting}
mkdir /srv/samba/ddgp

chmod 770 -R /srv/samba/ddgp

chown root.3000020 -R /srv/samba/ddgp
\end{lstlisting}

Todas as impressoras foram colocadas na rede, mapeadas no servidor do Samba 4 e compartilhadas para os demais computadores com a instalação dos drives automática.\\

\begin{lstlisting}
[printers] 

print ok = yes 

guest ok = yes

path = /var/spool/samba 

browseable = yes

[print$] 

path = /var/lib/samba/printers 

read only = yes

write list = root 

inherit permissions = yes
\end{lstlisting}

Tendo realizado todo este estudo com base na rede já existente da instituição de ensino foi constatado que a implementação sugerida neste trabalho é a mais adequada para atender os objetivos já explicitados anteriormente sobre otimização e segurança.