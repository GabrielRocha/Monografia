\appendix 

\chapter{Scripts}

\section{smbmanager.sh}

\#!/bin/bash

\#Gabriel Rocha

end=0

help="É NECESSÁRIO TER PERMISSÃO DE ROOT $\backslash$nUSO: smbmanager [OPCAO] [VALOR] $\backslash$n 
$\backslash$nOpções gerais:$\backslash$n -g [VALOR]   Grupo no qual será adicionado a máquina ou usuário  $\backslash$n -m [VALOR]   Nome da máquina a ser cadastrada $\backslash$n -u [VALOR]   Usuário a ser cadastrado no sistema e no samba $\backslash$n -d [VALOR]   Usuário a ser deletado do sistema $\backslash$n -x [VALOR]   Máquina a ser deletada do samba e do sistema"

AddMachine(){

if [ -n "\$machine" ] ; then

    if [ -z "\$group" ] ; then

        useradd "--disabled-login "--home /dev/null "--shell /bin/false \$machine$\backslash$\$ 2$ >$/dev/null \&\& passwd -l \$machine$\backslash$\$ \&\& smbpasswd -a -m \$machine

    fi

    if [ -n "\$group" ]; then
	
        useradd "--disabled-login "--home /dev/null "--shell /bin/false "--group \$group \$machine$\backslash$\$ 

	check=\$(echo \$$?$)

		if [ \$check -eq 0 ]; then
	
 			passwd -l \$machine$\backslash$\$ 2$>$/dev/null \&\& smbpasswd -a -m \$machine 
       fi

    fi        

fi

}

AddUser(){

if [ -n "\$user" ] ; then

    if [ -z "\$group" ] ; then

        adduser \$user 2$>$/dev/null 

        smbpasswd -a \$user

    fi

    if [ -n "\$group" ] ; then

        adduser \$user 2$>$/dev/null

        usermod -g \$user \$group

		check=\$(echo \$$?$)

		if [ \$check -eq 0 ]; then

        	smbpasswd -a \$user

		fi
		
    fi

fi

}

DelMachine(){

if [ -n "\$delmachine" ]; then    

    smbpasswd -x -m \$delmachine

    deluser \$delmachine$\backslash$\$

fi

}

DelUser(){

if [ -n "\$deluser" ]; then    

    smbpasswd -x \$deluser

    deluser \$deluser

fi

}

while getopts "hg:m:u:d:x:" paramentro;

do

   case \$paramentro in

     \ h) echo -e \$help;;

     \ g) group=\$OPTARG ;;

      m) machine=\$OPTARG ;;

      u) user=\$OPTARG ;;

      d) deluser=\$OPTARG ;;

      x) delmachine=\$OPTARG ;;

      *) echo -e \$help; end=1;;

   esac

done

if [[ "\$group" = *'-'* ]] $\|$ [[ "\$machine" = *'-'* ]] $\|$ [[ "\$user" = *'-'* ]] $\|$ [[ "\$deluser" = *'-'* ]] $\|$ [[ "\$delmachine" = *'-'* ]]; then

    echo -e \$help

else

    if [ \$end -ne 1 ] ; then

        AddMachine

        AddUser

        DelMachine

        DelUser

    fi

fi

\section{smbda.sh}
%Apêndices são opcionais, mas podem ser usados, por exemplo, para incluir tabelas com os dados brutos.

\#!/bin/sh
 
\#\#\#\#\#\#\#\#\#\#\#\#\#\#\#\#\#\#\#\#\#\#\#\#\#\#\#\#\#\#\#\#\#\#\#\#\#\#\#\#\#\#\#\#\#\#\#\#\#\#\#\#\#\#\#\#\#\#\#\#\#\#\#\#\#\#

\# Copyright (C) 2011 - Fabio Antonio Ferreira \hspace{160pt} \#

\# http://fantonio.wordpress.com | fantonios@gmail.com \hspace{106pt} \#

\# Este trabalho está licenciado sob uma Licença Creative Commons \hspace{59pt} \#

\# Atribuição-Compartilhamento pela mesma Licença 2.5 Brasil. Para ver a copia \#

\# desta licença, acesse: http://creativecommons.org/licenses/by-sa/2.5/br/ \hspace{34pt} \#

\# ou envie uma carta para Creative Commons, 171 Second Street, Suite 300, \hspace{19pt} \#

\# San Francisco, California 94105, USA. \hspace{188pt} \#

\# Modificações em 27 de Julho de 2012 por Gabriel Rocha (GBR) \hspace{67pt}             \#

\# email: gabriel.rocha.gbr@gmail.com \hspace{198pt} \#

\#\#\#\#\#\#\#\#\#\#\#\#\#\#\#\#\#\#\#\#\#\#\#\#\#\#\#\#\#\#\#\#\#\#\#\#\#\#\#\#\#\#\#\#\#\#\#\#\#\#\#\#\#\#\#\#\#\#\#\#\#\#\#\#\#\#

\# == FUNCOES ===============================================

USUARIO=`whoami`

if [ "\$USUARIO" != "root" ]; then

 	echo

  echo "======================================================"

  echo " ESTE PROGRAMA PRECISA SER EXECUTADO COM PERMISSOES DE SUPERUSUARIO!"  

  echo " Abortando..."

  echo "======================================================"

  echo

  exit 1

fi

\_HEAD () \{

`which clear`

echo "SISTEMA PARA ADICIONAR MAQUINA LINUX AO DOMÍNIO WINDOWS OU LINUX"

echo "========================================================="

\}

\_PACOTES () \{

        echo "Instalando os pacotes necessários";       


  apt-get install krb5-user libpam-krb5 winbind samba smbfs smbclient krb5-config libkrb53 libkdb5-4 libgssrpc4 -y $>$ /dev/null;
  
      check=\$(echo \$?)

        if [ \$check -eq 0 ]; then

           echo "Pacotes instalados com sucesso"

        else

           echo "Falha ao instalar os pacotes"

        fi

\}

\_HORA () \{

        echo "Atualizando data e hora";

        ntpdate br.pool.ntp.org $>$ /dev/null;

        echo "Horario atual:" `date`

        echo "Hora alterada com sucesso"

\}

\_BACKUP\_ORIG () \{

  \# Rotina de Backup dos arquivos de configurações.

	if [ ! -e /etc/krb5.conf\_backup ]; then

		cp /etc/krb5.conf /etc/krb5.conf\_backup $>$ /dev/null;

	fi

	if [ ! -e /etc/resolv.conf\_backup ]; then

		cp /etc/resolv.conf /etc/resolv.conf\_backup $>$ /dev/null

	fi

	if [ ! -e /etc/samba/smb.conf\_backup ]; then

        	cp /etc/samba/smb.conf /etc/samba/smb.conf\_backup $>$ /dev/null

	fi

	if [ ! -e /etc/nsswitch.conf\_backup ]; then

        	cp /etc/nsswitch.conf /etc/nsswitch.conf\_backup $>$ /dev/null

	fi

	if [ ! -e /etc/pam.d/common-account\_backup ]; then

	        cp /etc/pam.d/common-account /etc/pam.d/common-account\_backup $>$ /dev/null

	fi

	if [ ! -e /etc/pam.d/common-auth\_backup ]; then

	        cp /etc/pam.d/common-auth /etc/pam.d/common-auth\_backup $>$ /dev/null

	fi

	if [ ! -e /etc/pam.d/common-session\_backup ]; then

	        cp /etc/pam.d/common-session /etc/pam.d/common-session\_backup $>$ /dev/null

	fi

	if [ ! -e /etc/pam.d/sudo\_backup ]; then

	        cp /etc/pam.d/sudo /etc/pam.d/sudo\_backup $>$ /dev/null

	fi
         
        check=\$(echo \$?)

   if [ \$check -eq 0 ]; then

      echo "Rotina de Backup executada com sucesso!"

   else

      echo "Falha ao fazer o Backup."

   fi
         
\}

\_RETURN\_BACKUP () \{

        \# Rotina de Recuperação do Backup de configurações.

        mv /etc/krb5.conf\_backup /etc/krb5.conf $>$ /dev/null

        mv /etc/resolv.conf\_backup /etc/resolv.conf $>$ /dev/null

        mv /etc/samba/smb.conf\_backup /etc/samba/smb.conf $>$ /dev/null

        mv /etc/nsswitch.conf\_backup /etc/nsswitch.conf $>$ /dev/null

        mv /etc/pam.d/common-account\_backup /etc/pam.d/common-account $>$ /dev/null

        mv /etc/pam.d/common-auth\_backup /etc/pam.d/common-auth $>$ /dev/null
        
        mv /etc/pam.d/common-session\_backup /etc/pam.d/common-session $>$ /dev/null

        mv /etc/pam.d/sudo\_backup /etc/pam.d/sudo $>$ /dev/null
         
        
				check=\$(echo \$?)

   if [ \$check -eq 0 ]; then

      echo "Recuperação do Backup executada com sucesso!"

   else

      echo "Falha na recuperação do Backup."

   fi
         
\}

\_NOME\_DOMINIO () \{
 
   \#Entrada do nome do dominio ao qual deseja engreçar.

	 \#No caso do linux temos dois servidores um do KDC e outro do dominio

	 \#No windows informamos o servidor kdc

    read -p "Entre com o nome do Domínio:" var1

    dominio=\$(echo \$var1 | tr a-z A-Z)

    read -p "Entre com o seu KDC (key Distribution Center):" var2

    kdc=\$(echo \$var2 | tr A-Z a-z)         

\}

\_IP\_DNS (){

	\#IP do servidor de dns

	read -p "Entre com o IP do servidor de DNS:" ip

	echo "nameserver \$ip" $>$ /etc/resolv.conf

\}

\_SO\_SERVIDOR () \{

	\#Sistema Operacional do AD	

	read -p "Entre com o S.O. do servidor (Linux ou Windows): " so

	so=\$(echo \$so | tr a-z A-Z)

	workgroup=""

	if [ \$so = "LINUX" ] ; then

		read -p "Informe o Domain do Samba4: " workgroup

		workgroup=\$(echo \$workgroup | tr a-z A-Z)

	else

		workgroup=\$(echo \$var1)

	fi

\}

\_KRB5 () \{

   echo "[libdefaults]

   default\_realm = \$dominio

	 \# The following krb5.conf variables are only for MIT Kerberos.

      krb4\_config = /etc/krb.conf

      krb4\_realms = /etc/krb.realms

      kdc\_timesync = 1

      ccache\_type = 4

      forwardable = true

      proxiable = true

		\# The following libdefaults parameters are only for Heimdal Kerberos.

      v4\_instance\_resolve = false

      v4\_name\_convert = \{

           host = \{

               rcmd = host

               ftp = ftp

           \}  

           plain = \{

               something = something-else

           \}  

      \}  

      fcc-mit-ticketflags = true

   [realms]

   		\$dominio = \{

        	kdc = \$kdc
           
            admin\_server = \$kdc

           \}  
             
   [domain\_realm]

   		.\$var1 = \$kdc

   [login]

   		krb4\_convert = true

   		krb4\_get\_tickets = false" $>$ /etc/krb5.conf
 
   echo "Configuração alterada com sucesso!"

\}

\_TESTEAD () \{

   read -p "Entre com um usuário para testar sua conexão com o Active Directory:" user

   kinit \$user$@$\$dominio

    

   check=\$(echo \$?)

   if [ \$check -eq 0 ]; then

      echo "Sua máquina conectou com sucesso!"

   else

      echo "Falha ao se conectar com o Active Directory"

   fi

\}

\_SMB () \{

    

   maquina=\$(hostname)

   echo "\# Sample configuration file for the Samba suite for Debian GNU/Linux.

   \#======================= Global Settings =======================

   [global]

      workgroup = \$workgroup

      netbios name = \$maquina

      realm = \$var1

      server string = \% h Server

      dns proxy = no

  	  log file = /var/log/samba/log.\%m  

	  max log size = 1000

	  syslog = 0  

      panic action = /usr/share/samba/panic-action \%d

      security = ADS

      password server = \$kdc

      encrypt passwords = true

      passdb backend = tdbsam

      obey pam restrictions = yes

      unix password sync = yes

      passwd program = /usr/bin/passwd \%u
      
      pam password change = yes

      idmap uid = 10000-20000

      winbind gid = 10000-20000

      winbind enum users = yes

      winbind enum groups = yes

      winbind use default domain = yes

      template homedir = /home/\%D/\%U

      template shell = /bin/bash

   [homes]

      comment = Home Directories

      browseable = no

      read only = yes

      create mask = 0700

      directory mask = 0700

      valid users = \%S " $>$ /etc/samba/smb.conf

   echo "Configuração alterada com sucesso!"

\}

\_FUNC\_RESTART() \{

        \# Stop Winbind

        /etc/init.d/winbind stop $>$ /dev/null

        check=\$(echo \$?)

   if [ \$check -eq 0 ]; then

      echo "Winbind Stop!"

   else

      echo "Falha ao parar o Winbind"

   fi

     \# Restart Samba

     /etc/init.d/smbd restart $>$ /dev/null

     check=\$(echo \$?)

   if [ \$check -eq 0 ]; then

      echo "Samba restart com sucesso!"

   else

      echo "Falha no restart do Samba!"

   fi

    \# Start Winbind

    /etc/init.d/winbind start $>$ /dev/null

    check=\$(echo \$?)

   if [ \$check -eq 0 ]; then

      echo "Winbind start!"

   else

      echo "Falha ao fazer iniciar o Winbind!"

   fi

\}

\_ADDDOMINIO () \{
    
 
  echo "++++++++++++++++++++++++++++++++++++++++++++"

   echo "++  Adicionando a Máquina no Domínio  ++"

   echo "++++++++++++++++++++++++++++++++++++++++++++"

   \# Adicionando a máquina ao domínio

        read -p "Entre com um usuário administrador de Domínio:" user   

   net ads join -U \$user;

        check=\$(echo \$?)

        clear

        \# Validação da conexão com o domínio

        if [ \$check -eq 0 ]; then

      echo "Sua máquina foi adicionada no Domínio!"

   else

      echo "Falha ao adicionar a máquina no Domínio"

   fi

\}

\_TESTDOMINIO () \{

        \# Teste de requisição ao dominio

        wbinfo -t $>$ /dev/null

        check=\$(echo \$?)

   if [ \$check -eq 0 ]; then

      echo "Teste de Domínio!"

   else

      echo "Falha ao testar o Domínio"

   fi

\}

\_FUNCAUTENTICACAO () \{

        \# Configurando o arquivo nsswitch.conf

        echo "passwd:         compat winbind

              group:          compat winbind

              shadow:         compat" $>$ /etc/nsswitch.conf

        \# Teste de configuração do Winbind        

        check=\$(echo \$?)
   		
		if [ \$check -eq 0 ]; then

      echo "Winbind testado com sucesso!"

   else

      echo "Falha ao testar o Winbind"

   fi

        \# PAM - common-account

        echo "account sufficient       pam\_winbind.so
              account required         pam\_unix.so" $>$ /etc/pam.d/common-account

        \# PAM - common-auth

        echo "auth sufficient pam\_winbind.so

              auth sufficient pam\_unix.so nullok\_secure use\_first\_pass

              auth required   pam\_deny.so" $>$ /etc/pam.d/common-auth

        \# PAM - common-session      

        echo "session required pam\_unix.so

              session required pam\_mkhomedir.so umask=0022 skel=/etc/skel" $>$ /etc/pam.d/common-session

        \# PAM - sudo

        echo "auth sufficient pam\_winbind.so

              auth sufficient pam\_unix.so use\_first\_pass

              auth required   pam\_deny.so

              $@$include common-account" $>$ /etc/pam.d/sudo

        \# Teste de configuração do PAM

        check=\$(echo \$?)

   if [ \$check -eq 0 ]; then

      echo "PAM configurado com sucesso!"

   else

      echo "Falha ao configurar o PAM"

   fi

\}

\_FUNC\_HOMEDIR () \{

        HOME\_DIR=\$var1

        if [ -d /home/\$HOME\_DIR ]; then

                echo "Já existe este diretório !"                

        else

                echo "Este diretório não existe !"

                echo "Criando o diretório \$HOME\_DIR"

      mkdir /home/\$var1

                sleep 2

        fi

\}

\_FUNC\_DEL\_MAQ\_DOMINIO () \{

    

   maquina=\$(hostname)

        echo "++++++++++++++++++++++++++++++++++++++++++++"

        echo "++  Removendo a Máquina no Domínio  ++"

        echo "++++++++++++++++++++++++++++++++++++++++++++"
       
        \# Remover a máquina ao domínio

        read -p "Entre com um usuário administrador de Domínio:" user

   net ads status -U \$user

   check1=\$(echo \$?)   

   clear

   \# Validação se a máquina está no domínio

   if [ \$check1 -eq 255 ]; then

      echo "A máquina \$maquina não está no dominio"

   else

      \# Validação de remoção de máquina do domínio

      net ads leave -U \$user;

      check=\$(echo \$?)

      clear

      if [ \$check -eq 0 ]; then

         echo "Sua máquina foi removida do Domínio!"

      else

         echo "Falha ao remover a máquina no Domínio"

      fi

   fi

\}

\# =========================================================

\# Menu de seleção

echo "Linux Active Directory:"

echo "(1) Adicionar Máquina no Domínio"

echo "(2) Remover Máquina do Domínio"

echo "(3) Verificar conexão com o Domínio"

echo "(0) Sair"

echo "Digite a opção desejada:"

read resposta

case "\$resposta" in

        1)  

      \_HEAD

      \_PACOTES

      \_HORA

      \_BACKUP\_ORIG

      \_NOME\_DOMINIO

      \_IP\_DNS

      \_SO\_SERVIDOR

      \_KRB5

      \_TESTEAD

      \_SMB

      \_FUNC\_RESTART

      \_ADDDOMINIO

      \_TESTDOMINIO

      \_FUNCAUTENTICACAO

      \_FUNC\_RESTART

      echo "++++++++++++++++++++++++++++++++++++++++++++"

      echo "++ Bem vindo ao dominio \$dominio ++"

      echo "++++++++++++++++++++++++++++++++++++++++++++"

                ;;  

        2)  

       \_FUNC\_DEL\_MAQ\_DOMINIO

		\_RETURN\_BACKUP

                ;;  

        3)  

       \_TESTDOMINIO

                ;;  

        0)  

                exit

                ;;  

        \*)  

                echo 'Opção Inválida!'

esac
