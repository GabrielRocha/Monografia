\chapter{CONCEITOS E TÉCNICAS NECESSÁRIAS}

O capitulo explica termos técnicos essenciais para o melhor entendimento do trabalho.

\section{Samba}

/* Samba é um software open source e reimplementa os protocolos SMB e CIFS para prover
uma série de serviços para ambiente Windows, como servidor de arquivos e impressão e pode
ser usado em um Servidor de Domínio como um Primary Domain Controller (PDC) ou como
como um membro do domínio, e pode também ser usado como parte de um domínio Active
Directory. */

Samba é um software open source que provê serviços a clientes nos protocolos SMB e CIFS.
O samba permite a interoperabilidade entre servidores Linux/Unix e clientes baseados na
plataforma Windows.
O samba permite que um servidor linux seja apto a fornecer serviços como:
  \begin{itemize}
    \item \textbf{\#Servidor de arquivos e impressão}
    \item \textbf{\#Autenticação e autorização}
    \item \textbf{\#Resolução e busca de nomes}
    \item \textbf{\#Servidor de domínio como PDC}
  \end{itemize}

\section{Permissões no Linux}

\section{Seções}

\section{Parâmetro}

\section{Variáveis}

\section{Variáveis Especiais do Samba}

\section{PDC}

\section{Comandos Básicos do Samba3}

\section{SAMBA-TOOLS}

\section{SMBD}

\section{NMDB}

\section{NETBIOS}

\section{Domain Master}

DOMAIN MASTER BROWSER Uma vez que o Local Master Browser é eleito no segmento de rede, uma consulta é feita ao servidor WINS para saber quem é o Domain Master Browser da rede para enviar a lista de compartilhamentos. A máquina escolhida como Local Master Browser envia pacotes para a porta UDP 138 do Domain Master e este responde pedindo a lista de todos os nomes de máquinas que o Local Master conhece e também o registra como Local Master para aquele segmento de rede.

\section{Master Browser}

\section{WINS}

\section{BIND}

\section{Ldap}

\section{Kerberos}

\section{NTVFS}

Sistema de arquivos que armazena os atributos do NTFS

\section{Referencias - Temporário}
SAMBA:
http://pt.wikipedia.org/wiki/Samba_(servidor)
http://en.wikipedia.org/wiki/Samba_(software)
http://www.samba.org/samba/docs/
http://www.samba.org/samba/what_is_samba.html
http://www.samba.org/samba/docs/SambaIntro.html
http://www.samba.org/cifs/docs/what-is-smb.html
http://www.samba.org/cifs/
