\chapter{CONCEITOS E TÉCNICAS NECESSÁRIAS}

O capitulo explica termos técnicos essenciais para o melhor entendimento do trabalho.

\section{Samba}

/* Samba é um software open source e reimplementa os protocolos SMB e CIFS para prover
uma série de serviços para ambiente Windows, como servidor de arquivos e impressão e pode
ser usado em um Servidor de Domínio como um Primary Domain Controller (PDC) ou como
como um membro do domínio, e pode também ser usado como parte de um domínio Active
Directory. */

Samba é um software open source que provê serviços a clientes nos protocolos SMB e CIFS.
O samba permite a interoperabilidade entre servidores Linux/Unix e clientes baseados na
plataforma Windows.
O samba permite que um servidor linux seja apto a fornecer serviços como:
  \begin{itemize}
    \item \textbf{\#Servidor de arquivos e impressão} Utilizando o protocolo Server Message Block para possibilitar o compartilhamento de arquivos, pastas volumes e impressoras na rede.

    \item \textbf{\#Autenticação e autorização} Identifica um computador ou um usuário da rede e determina os direitos de acesso a arquivos que cada usuário possui, através de tecnologias como permissões de arquivos, diretivas de grupo e o serviço de autenticação Kerberos.

    \item \textbf{\#Resolução e busca de nomes e diretórios} Compartilha as principais informações sobre computadores e usuários da rede através do Light Directory Access Protocol (LDAP) e o Microsoft Active Directory.

    \item \textbf{\#Servidor de domínio como PDC} Funcionando como controlador de domínio ativo dentro de um domínio Windows.
  \end{itemize}

\section{Permissões no Linux}

\section{Permissões especiais no Linux}

Existe no Linux três permissões especiais, para dar segurança ao sistema, chamadas assim por somente serem atribuídas a arquivos específicos (arquivos executáveis e diretórios). Tais permissões são fornecidas pelos bits SUID, SGID e STICKY.

  \begin{itemize}
    \item \textbf{\#SUID} O bit SUID (Set UID) é aplicável apenas a arquivos executáveis, fazendo com que estes rodem com as permissões de seu proprietário, independente de quem tenha executado-o. Pode ser útil para que usuários comuns possam executar arquivos permitidos apenas a administradores.

    \item \textbf{\#SGID} O bit SGID (Set GID) pode ser aplicado a um arquivo executável e a um diretório. No primeiro caso ele tem as mesma função do SUID, porém rodando com as permissões de um grupo de usuários. No segundo, ele força os arquivos e diretórios criados dentro do diretório pai (o que obteve a permissão) a pertencerem ao mesmo grupo, independente do grupo de quem tenha-os criado.

    \item \textbf{\#STICKY} O bit STICKY é aplicável a diretórios e faz com que a exclusão de arquivos pertencentes a estes diretórios seja apenas permitida ao dono do arquivo e ao administrador do sistema. Tem vantagem sobre a permissão “Somente Leitura” no diretório pois faz com que outros usuários possam criar e editar qualquer arquivo, impedindo-os apenas de apagá-lo.
  \end{itemize}

\section{Seções}

No Samba, as configurações de compartilhamentos, configurações de impressoras e todas as configurações gerais, são realizadas através de um unico arquivo de configuraçõa, o "/etc/samba/smb.conf". Esse arquivo para melhor organização, fica dividio em sessões, sendo a primeira sessão nomeada como [global], onde são definidas as configurações gerais do servidor. Também podem ser criadas sessões adicionais para cada compartilhamento, sendo nomeadas com o nome do mesmo. Se desejamos criar um compartilhamento com o nome "arquivo", a sessão que deve ser criada no arquivo de configuração deve ser [arquivo].

\section{Parâmetro}

\section{Variáveis}

\section{Variáveis Especiais do Samba}

\section{PDC}

\section{Comandos Básicos do Samba3}

\section{SMBD}

É um daemon que permite compartilhamento de arquivos e impressoras em uma rede SMB e provê autorização e autenticação a usuários SMB. O SMBD é um dos componentes principais do Samba, e 

\section{NMBD}

É um daemon que cuida do Windows Internet Name Service (WINS) e auxilia com a navegação e resolução de nomes.

\section{NETBIOS}

NETBIOS, Networking Basic Input/Outbut System, é uma API desenvolvida em 1984 pela IBM, que fornece serviços relacionados na camada de sessão do modelo OSI, permitindo a comunicação entre computadores na rede através de um nome NETBIOS correspondente a um hostname.

\section{Domain Master}

***VERIFICAR REFERENCIA***

DOMAIN MASTER BROWSER Uma vez que o Local Master Browser é eleito no segmento de rede, uma consulta é feita ao servidor WINS para saber quem é o Domain Master Browser da rede para enviar a lista de compartilhamentos. A máquina escolhida como Local Master Browser envia pacotes para a porta UDP 138 do Domain Master e este responde pedindo a lista de todos os nomes de máquinas que o Local Master conhece e também o registra como Local Master para aquele segmento de rede.

\section{Master Browser}

\section{WINS}

Windows Internet Name Service (WINS) ou NetBIOS Name Service (NBNS) é um serviço do protocolo TCP/IP. Este serviço faz a resolução nomes e números IP e os armazena, disponibilizando esta informação para quem necessite usar. Cada cliente envia seu nome NetBIOS e número IP para o servidor WINS, que armazena estas informações em um banco de dados. Quando um cliente desejar se comunicar com um outro, ele envia o nome desejado ao servidor WINS. Se o nome constar na base de dados, o servidor WINS retorna ao solicitante o número IP.

\section{BIND}

\section{Ldap}

O LDAP é o protocolo responsável por fornecer Serviço de Diretórios a computadores Windows de forma similar ao Active Directory da Microsoft, que é baseado no LDAP. Tais serviços incluem conexões de computadores, grupos de computadores, usuários, administração de identidades, além de possibilitar uma maneira eficiente de descrever, localizar e administrar esses recursos.

\section{Kerberos}

Kerberos é um protocolo de segurança de rede e fornece autenticação entre conputadores e usuários através de um servidor centralizado que concede autenticações criptograficas a qualquer computador utilizando o Kerberos. Esse sistema de segurança e autenticação agraga diversos benefícios como autentificação mútua, autentificação delegada, interoperabilidade e gerência simplificada e confiável. O samba pode usar o Kerberos como um mecanismo autenticação de computadores e usuários.

\section{NTVFS}

Sistema de arquivos que armazena os atributos do NTFS

\section{GSSAPI}

A GSSAPI é uma interface que permite desenvolvedores escreverem aplicações que aproveitam mecanismos de segurança tais como Kerberos, sem ter de programar explicitamente para qualquer mecanismo, ou seja, aplicações genéricas do ponto de vista de segurança. Programas que usam GSSAPI são, deste modo, altamente portáteis, não somente de uma plataforma para outra, mas de uma configuração de segurança a outra e de um protocolo de transporte a outro. A GSSAPI fornece vários níveis de proteção de dados, consistentes com os mecanismos de segurança subjacentes.


\section{Referencias - Temporário}
SAMBA:
http://pt.wikipedia.org/wiki/Samba\_(servidor)

http://en.wikipedia.org/wiki/Samba\_(software)

http://www.samba.org/samba/docs/

http://www.samba.org/samba/what\_is\_samba.html

http://www.samba.org/samba/docs/SambaIntro.html

http://www.samba.org/cifs/docs/what\-is\-smb.html

http://www.samba.org/cifs/

Using Samba (OREILLY)

NETBIOS:

http://pt.wikipedia.org/wiki/Netbios

Using Samba (OREILLY)

SMBD: 

Using Samba (OREILLY)
http://www.samba.org/samba/docs/

NMBD:

Using Samba (OREILLY)
http://www.samba.org/samba/docs/

GSSAPI:
http://www.gta.ufrj.br/grad/10\_1/kerberos/gssapi.html