\chapter{CONCEITOS E TÉCNICAS NECESSÁRIAS}

O capitulo explica termos técnicos essenciais para o melhor entendimento do trabalho.

\section{Samba}

/* Samba é um software open source e reimplementa os protocolos SMB e CIFS para prover
uma série de serviços para ambiente Windows, como servidor de arquivos e impressão e pode
ser usado em um Servidor de Domínio como um Primary Domain Controller (PDC) ou como
como um membro do domínio, e pode também ser usado como parte de um domínio Active
Directory. */

Samba é um software open source que provê serviços a clientes nos protocolos SMB e CIFS.
O samba permite a interoperabilidade entre servidores Linux/Unix e clientes baseados na
plataforma Windows.
O samba permite que um servidor linux seja apto a fornecer serviços como:
  \begin{itemize}
    \item \textbf{\#Servidor de arquivos e impressão} Utilizando o protocolo Server Message Block para possibilitar o compartilhamento de arquivos, pastas volumes e impressoras na rede.
    \item \textbf{\#Autenticação e autorização}Identifica um computador ou um usuário da rede e determina os direitos de acesso a arquivos que cada usuário possui, através de tecnologias como permissões de arquivos, diretivas de grupo e o serviço de autenticação Kerberos.
    \item \textbf{\#Resolução e busca de nomes e diretórios}Compartilha as principais informações sobre computadores e usuários da rede através do Light Directory Access Protocol (LDAP) e o Microsoft Active Directory.
    \item \textbf{\#Servidor de domínio como PDC}Funcionando como controlador de domínio ativo dentro de um domínio Windows.
  \end{itemize}

\section{Permissões no Linux}

\section{Permissões especiais no Linux}

\section{Seções}

\section{Parâmetro}

\section{Variáveis}

\section{Variáveis Especiais do Samba}

\section{PDC}

\section{Comandos Básicos do Samba3}

\section{SAMBA-TOOLS}

\section{SMBD}

\section{NMDB}

\section{NETBIOS}

\section{Domain Master}

DOMAIN MASTER BROWSER Uma vez que o Local Master Browser é eleito no segmento de rede, uma consulta é feita ao servidor WINS para saber quem é o Domain Master Browser da rede para enviar a lista de compartilhamentos. A máquina escolhida como Local Master Browser envia pacotes para a porta UDP 138 do Domain Master e este responde pedindo a lista de todos os nomes de máquinas que o Local Master conhece e também o registra como Local Master para aquele segmento de rede.

\section{Master Browser}

\section{WINS}

\section{BIND}

\section{Ldap}

\section{Kerberos}

\section{NTVFS}

Sistema de arquivos que armazena os atributos do NTFS

\section{Referencias - Temporário}
SAMBA:
http://pt.wikipedia.org/wiki/Samba\_(servidor)

http://en.wikipedia.org/wiki/Samba\_(software)

http://www.samba.org/samba/docs/

http://www.samba.org/samba/what\_is\_samba.html

http://www.samba.org/samba/docs/SambaIntro.html

http://www.samba.org/cifs/docs/what\-is\-smb.html

http://www.samba.org/cifs/

Using Samba (OREILLY)

\section{GSSAPI}

A GSSAPI é uma interface que permite desenvolvedores escreverem aplicações que aproveitam mecanismos de segurança tais como Kerberos, sem ter de programar explicitamente para qualquer mecanismo, ou seja, aplicações genéricas do ponto de vista de segurança. Programas que usam GSSAPI são, deste modo, altamente portáteis, não somente de uma plataforma para outra, mas de uma configuração de segurança a outra e de um protocolo de transporte a outro. A GSSAPI fornece vários níveis de proteção de dados, consistentes com os mecanismos de segurança subjacentes. (http://www.gta.ufrj.br/grad/10\_1/kerberos/gssapi.html)