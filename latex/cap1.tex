\chapter{INTRODUÇÃO}

A organização lógica de uma rede é uma necessidade que se torna bem visível quando se possui um cenário com muitos computadores e muitos usuários, e a implementação de um domínio se torna indispensável para o gerênciamento mais eficiente desse tipo de recurso. A implementação de um domínio na rede permite que se possa fornecer maior confiabilidade, facilidade de manutenção, controle de recursos e aumento da segurança da informação.
Esses recursos de gerenciamento de domínios são fornecidos através de PDC (Primary Domain Controller) e estão disponiveis para os sistemas operacionais mais atuantes do mercado, sendo que o principal e mais utilizado é o da Microsoft conhecido como AD (Active Directory).
Devido ao fato de ser um software proprietário, a solução da Microsoft acaba não sendo a melhor alternativa para muitas organizações devido seu alto custo de utilização e manutenção, e por não permitir a customização dessas ferramentas para que atendam melhor as necessidades de cada cenário distinto.
Recorrendo a ferramentas de software livre como o Samba 3 e o Samba 4 para obter a mesma solução, as organizações podem ter sem custo os recursos necessários para implementar e gerenciar um domínio em suas redes, porém há uma grande dificuldade devido ao fato de essas ferramentas livres ainda não estarem maduras o suficiente para a fácil utilização, mas apesar do baixo nível de maturidade, já se mostram capazes de substituir integralmente as ferramentas proprietárias existentes.

\section{Objetivo}

O objetivo deste trabalho é apresentar tecnologias para o gerenciamento de recursos de redes que estão em fase de desenvolvimento e amadurecimento e que demonstram ser promissoras e se tornaram excelentes alternativas às principais ferramentas proprietárias existentes no mercado, além de servir como base para estudo de servidores linux e implementar, através de um estudo de caso, um serviço que pretende melhorar o controle da rede no IFF – campus Bom Jesus, e também melhorar e proporcionar maior segurança digital e diminuir o tempo de manutenção dos incidentes.

\section{Justificativa do trabalho}

A implementação de um servidor de domínio no IFF – Campus Bom Jesus possibilitará um maior controle dos usuários que acessam o sistema, e assim será possível saber quem está logado no sistema, permitir ou bloquear o acesso à pastas e compartilhamentos pela rede, realizar a substituição mais fácil e ágil de equipamentos sem ter a necessidade do usuário ficar esperando a manutenção da máquina.

O servidor de impressão permite que todas as impressoras sejam mapeadas por setor possibilitando que mais de uma máquina possa imprimir no mesmo equipamento sem ter uma conexão física entre elas.

O servidor de arquivos permite a centralização e compartilhamento de arquivos através da rede, além de aumentar a confiabilidade do armazenamento desses arquivos, possibilitanto o fácil backup dos dados, e permitindo controle de acesso das informações. Da mesma forma, é possivel criar perfis móveis, onde os dados e arquivos dos usuários ficam armazenados no servidor, possibilitanto o acesso desses perfis em qualquer computador ligado na rede.

\section{Estrutura do trabalho}

Este trabalho está dividido em seis capítulos, dispostos da seguinte forma:

O primeiro capítulo contém a introdução do trabalho descrevendo o problema identificado, o objetivo da implantação da proposta aqui abordada e a justificativa para a mesma.

O segundo capítulo apresenta uma breve explicação sobre as ferramentas e os termos técnicos utilizados para a implementação que é objetivo deste trabalho. 

O terceiro capítulo descreve um passo-a-passo para instalação e configuração do servidor Samba 3, desde o momento do download do pacote até o cadastro de usuários e máquinas e a integração com o Windows e Linux.

No quarto capítulo é apresentado um passo-a-passo similar ao do terceiro capítulo, porém utilizando a versão 4 do Samba.

O quinto capítulo apresenta um estudo de caso descrevendo a estrutura da instituição tida como proposta para a implementação do servidor abordado neste trabalho.

O sexto capítulo apresenta as conclusões tiradas do estudo, além dos trabalhos futuros que poderão ser realizados a partir deste.

Além dos capítulos descritos acima há uma área destinada aos scripts utilizados nas configurações necessárias.