\chapter{INTRODUÇÃO}

A organização lógica de uma rede é uma necessidade que se torna bem visível quando se possui um cenário com 110 computadores administrativos e entorno de 140 usuários, entre servidores e alunos bolsistas, e a implantação de um domínio se torna indispensável para o gerenciamento mais eficiente desse tipo de recurso. A implantação de um domínio na rede permite que se possa fornecer maior confiabilidade, facilidade de manutenção, controle de recursos e aumento da segurança da informação.

Esses recursos de gerenciamento de domínios são fornecidos através de PDC (\textit{Primary Domain Controller}) e estão disponíveis para os sistemas operacionais mais atuantes do mercado, sendo o AD (\textit{Active Directory}) da Microsoft considerada pelo sensso comum a principal e mais utilizada ferramenta para essa finalidade. Pelo fato de ser um \textit{software} proprietário, a solução da Microsoft acaba não sendo a melhor alternativa para muitas organizações pois possui alto custo de utilização e manutenção, e por não permitir a customização dessas ferramentas para que atendam melhor as necessidades de cada cenário distinto.

Recorrendo a ferramentas de \textit{software} livre como o Samba 3 e 4 para obter a mesma solução, as organizações podem ter sem custo os recursos necessários para implantar e gerenciar um domínio em suas redes. Porém, percebe-se uma grande dificuldade na configuração dessas ferramentas, principalmente para usuários não “acostumados” ao Linux, devido ao fato delas apresentarem o terminal de comando e arquivos texto como as principais formas de configuração, não necessitando de interface gráfica para tal. Entretanto, após configurado, o Samba 3 e 4 se mostram capazes de substituir integralmente as ferramentas proprietárias existentes.

\section{Objetivo}

O objetivo deste trabalho é apresentar especificamente a implantação do \textit{software} Samba, em suas versões 3 e 4, como tecnologias livres e viáveis para o gerenciamento de alguns recursos de rede, como autenticação, compartilhamento de arquivos e impressoras. Além de servir como base para estudo de servidores Linux, este trabalho visa implantar, através de um estudo de caso, um serviço que pretende melhorar o controle da rede no Instituto Federal Fluminense (IFF), especificamente do campus Bom Jesus do Itabapoana, proporcionando maior segurança digital e diminuindo o tempo de manutenção dos incidentes.
%Diferentemente do Samba 3 que já está estável há vários anos, o Samba 4 ainda esta em fase de desenvolvimento e amadurecimento, mas já demonstra ser promissor e possivelmente se tornará uma excelente alternativa às principais ferramentas proprietárias existentes no mercado. 

\section{Justificativa do trabalho}

A implantação de um servidor de domínio no IFF – Campus Bom Jesus do Itabapoana possibilitará um maior controle dos usuários que acessam o sistema, e assim será possível saber quem está logado, permitir ou bloquear o acesso à pastas e compartilhamentos pela rede, realizar a substituição mais fácil e ágil de equipamentos sem ter a necessidade do usuário ficar esperando a manutenção da máquina.

O servidor de impressão permitirá que todas as impressoras sejam mapeadas por setor possibilitando que mais de uma máquina possa imprimir no mesmo equipamento sem ter uma conexão física entre elas.

O servidor de arquivos permitirá a centralização e compartilhamento de arquivos através da rede, além de aumentar a confiabilidade do armazenamento desses arquivos, possibilitando o fácil \textit{backup} dos dados, e permitindo controle de acesso das informações. Da mesma forma, é possível criar perfis móveis, onde os dados e arquivos dos usuários ficam armazenados no servidor, possibilitando o acesso desses perfis em qualquer computador ligado na rede. Será possível também, embora não seja o foco deste trabalho, a utilização de programas de cotas de impressão, possibilitando definir a quantidade de páginas que um determinado usuário/setor poderá imprimir em um determinado espaço de tempo (diário, semanal, mensal, etc), ou simplesmente fazer o monitoramento das impressões realizadas nas impressoras compartilhadas pelo servidor.

Cabe ressaltar que a escolha do Samba para o campus foi motivada não só visando atender aos aspectos econômicos e legais da Instrução Normativa Nº 1, de 17 de janeiro de 2011, mas também pela história do projeto, pelo desempenho e estabilidade do \textit{software}, por ser um \textit{software} livre, garantindo assim as liberdades do usuário, no caso do campus e da equipe de TI (Tecnologia da Informação) em questão, além da grande comunidade entorno do projeto, que se mostra bastante ativa, disponibilizando bom suporte e grande quantidade de informações através de fóruns e sites especializados.

\section{Estrutura do trabalho}

Este trabalho está dividido em seis capítulos, incluindo a presente introdução. Os demais capítulos estão dispostos da seguinte forma:

O segundo capítulo apresenta uma breve explicação sobre as ferramentas e os termos técnicos utilizados para a implantação que é objetivo deste trabalho. 

O terceiro capítulo descreve um passo-a-passo para instalação e configuração do servidor Samba 3, desde o momento do \textit{download} do pacote até o cadastro de usuários, máquinas e a integração com o Windows e Linux.

No quarto capítulo é apresentado um passo-a-passo similar ao do terceiro capítulo, porém utilizando a versão 4 do Samba.

O quinto capítulo apresenta um estudo de caso descrevendo a estrutura da instituição tida como proposta para a implantação do servidor abordado neste trabalho.

O sexto capítulo apresenta as conclusões do estudo, além dos trabalhos futuros que poderão ser realizados a partir deste.

Além dos capítulos descritos acima há uma área destinada aos \textit{scripts} utilizados nas configurações necessárias.