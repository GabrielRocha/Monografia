\chapter{INTRODUÇÃO}

TEXTO

\section{Justificativa do trabalho}

A implementação de um servidor de domínio no IFF – Campus Bom Jesus possibilitará um maior controle dos usuários que acessam o sistema, e assim será possível saber quem está logado no sistema, permitir ou bloquear o acesso à pastas e compartilhamentos pela rede, realizar a substituição mais fácil e ágil de equipamentos sem ter a necessidade do usuário ficar esperando a manutenção da máquina.

O servidor de impressão permite que todas as impressoras sejam mapeadas por setor possibilitando que mais de uma máquina possa imprimir no mesmo equipamento sem ter uma conexão física entre elas.

\section{Objetivo}

O foco deste trabalho é servir como base para estudo de servidores linux e implementar um serviço que busca melhorar o controle da rede no IFF – campus Bom Jesus, e também melhorar e proporcionar maior segurança digital e diminuir o tempo de manutenção dos incidentes.

\section{Estrutura do trabalho}

Este trabalho está dividido em seis capítulos, dispostos da seguinte forma:

O primeiro capítulo contém a introdução do trabalho descrevendo o problema identificado, o objetivo da implantação da proposta aqui abordada e a justificativa para a mesma.

O segundo capítulo apresenta uma breve explicação sobre as ferramentas e os termos técnicos utilizados para a implementação que é objetivo deste trabalho. 

O terceiro capítulo descreve um passo-a-passo para instalação e configuração do servidor Samba 3, desde o momento do download do pacote até o cadastro de usuários e máquinas e a integração com o Windows e Linux.

No quarto capítulo é apresentado um passo-a-passo similar ao do terceiro capítulo, porém utilizando a versão 4 do Samba.

O quinto capítulo apresenta um estudo de caso descrevendo a estrutura da instituição tida como proposta para a implementação do servidor abordado neste trabalho.

O sexto capítulo apresenta as conclusões tiradas do estudo, além dos trabalhos futuros que poderão ser realizados a partir deste.

Além dos capítulos descritos acima há uma área destinada aos scripts utilizados nas configurações necessárias.