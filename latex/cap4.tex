\chapter{SAMBA 4}

O samba 4 vem com a proposta de criar um Active Directory livre, utilizando o LDAP, Bind e Kerberos.

\section{Instalação do SAMBA4}

Por se tratar de um sistema ainda em fase de produção alguns erros podem aparecer ou alguns parâmetros deverão ser modificados.
A instalação é realizada a partir do terminal, mas antes é necessário a instalação de algumas bibliotecas.

\# apt-get install build-essential libattr1-dev libblkid-dev libgnutls-dev python-dev autoconf python-dnspython git-core

Antes de começar a instalação o relógio do servidor tem que estar atualizado. O comando ntpdate deve ser usado para esse fim, onde um dos principais servidores é o br.pool.ntp.org.

\# ntpdate br.pool.ntp.org


%\begin{itemize}
	%\item \textbf{\# ntpdate br.pool.ntp.org} - Atualiza a hora do servidor a partir do servidor br.pool.ntp.org.
%\end{itemize}

O código fonte esta hospedado no servidor git dos desenvolvedores do samba, e o mesmo deve ser clonado para a maquina de destino.

\# git clone git://git.samba.org/samba.git samba-master; cd samba-master

O samba 4 segue os procedimento padrões de instalação de aplicativos no linux através do terminal, que segundo (???) se segue com o ./configure, make e o make install.
Explicação dos procedimentos.
Nesse caso ao invés de se utilizar o ./configure como padrão é utilizado o ./configure.developer, pois o mesmo habilita alguns modos de debug.

Para verificar a versão instalada é só executar o seguinte comando:

\# /usr/local/samba/bin/smbclient "--version 

\section{Criação de Domínio com o Samba 4}

Por padrão o samba 4 é instalado no /usr/local/.

\# cd /usr/local/samba

A instalação é a partir da execução do comando provision que fica localizado no /sbin do samba e a inserção de alguns parâmetros.

\# sbin/provision "--use-ntvfs "--realm=NOME\_SERVIDOR "--domain=NOME\_DOMINIO  "--adminpass= Senha12 "--server-role='domain controller'

\begin{enumerate}
	\item \textbf{use-ntvfs} - Habilita o NTVFS;
	\item \textbf{realm} - Domínio do servidor Kerberos;
	\item \textbf{domain} - Domínio do samba;
	\item \textbf{adminpass} - Senha do Administrator, essa senha deve ter pelo menos uma letra maiúscula;
	\item \textbf{server-role} - Regra do servidor.
\end{enumerate}

Depois de instalado e configurado o servidor de Active Directory pode ser iniciado. Uma das forma é inicia-lo em modo debug para poder acompanhar melhor os processos realizados.

\# /usr/local/samba/sbin/samba -i -M single

%\begin{itemize}
%	\item \textbf{\# apt-get install build-essential libattr1-dev libblkid-dev libgnutls-dev python-dev autoconf python-dnspython git-core} - Pacotes necessários para a compilação do samba 4 e download;
%	\item \textbf{\# git clone git://git.samba.org/samba.git samba-master; cd samba-master} - Faz um clone do samba 4 que esta no repositório para o servidor;
%	\item \textbf{\# ./configure.developer} - Configurar as bibliotecas com parâmetros de desenvolvedor. Habilitando alguns modos de debug;
%	\item \textbf{\# make} - Faz uma leitura do comando Makefile;
%	\item \textbf{\# make install} - Executa os comandos configurados para o parâmetro install do arquivo Makefile;
%	\item \textbf{\# /usr/local/samba/sbin/provision "--use-ntvfs "--realm=iff.bomjesus "--domain=iff  "--adminpass= Senha00 "--server-role='domain controller'} - Cria o domínio samba com AD;
%		\begin{enumerate}
%			\item \textbf{use-ntvfs} - Habilita o NTVFS;
%			\item \textbf{realm} - Domínio do servidor Kerberos;
%			\item \textbf{domain} - Domínio do samba;
%			\item \textbf{adminpass} - Senha do Administrator, essa senha deve ter pelo menos uma letra maiúscula;
%			\item \textbf{server-role} - Regra do servidor.
%		\end{enumerate}
%	\item \textbf{\# /usr/local/samba/bin/smbclient "--version} - Mostra a versão do samba;
%	\item \textbf{\# /usr/local/samba/sbin/samba -i -M single} - Inicia o samba 4 com o modo debug;
%	\item \textbf{\#  echo 'ip do servidor iff.bomjesus iff' $>$$>$ /etc/hosts} - Define um nome para o ip do servidor.
%\end{itemize}

\section{Instalação e configuração do BIND9}

BIND (Berkeley Internet Name Domain ou, como chamado previamente, Berkeley Internet Name Daemon[1]) é o servidor para o protocolo DNS mais utilizado na Internet,[2] especialmente em sistemas do tipo Unix, onde ele pode ser considerado um padrão de facto. Foi criado por quatro estudantes de graduação, membros de um grupo de pesquisas em ciência da computação da Universidade de Berkeley, e foi distribuído pela primeira vez com o sistema operacional 4.3BSD. O programador Paul Vixie, enquanto trabalhava para a empresa DEC, foi o primeiro mantenedor do BIND. Atualmente o BIND é suportado e mantido pelo Internet Systems Consortium.
Para a versão 9, o BIND foi praticamente reescrito. Ele passou a suportar, dentre outras funcionalidades, a extensão DNSSEC e os protocolos TSIG e IPv6.(http://pt.wikipedia.org/wiki/BIND)

O samba 4 já vem pré configurado para trabalhar com BIND9 para ser o servidor DNS nas versões 9.8 e 9.9.
Atualmente a versão do Bind9 no repositório é a 9.7 e com isso é gerada algumas incompatibilidades e para resolver esses problemas é feita o download e a  instalação manual da versão 9.9.

\# wget ftp://ftp.isc.org/isc/bind9/9.9.0/bind-9.9.0.tar.gz

Descompactazação do pacote baixado.
 
\# tar -xzvf bind-9.9.0.tar.gz

Entrar no diretório do bind9

\# cd bind-9.9.0

Configuração para a instalação, informando qual o local de instalação e onde ficarão os arquivos de configuração.

\# ./configure "--prefix=/usr/local/bind9 "--sysconfdir=/etc/bind

*****EXPLICAR O MAKE*****

*****EXPLICAR MAKE INSTALL*****

Entrar no diretório onde se encontra os arquivos de configuração do bind

\# cd /etc/bind

Com esse procedimento de instalação os arquivos de configuração não são gerados automaticamente, com isso gerando a necessidade de cria-los manualmente.

\# vim named.conf.options

As seguintes configurações devem ser adicionadas.

options \{
	
directory "/usr/local/bind/var/run/named";

tkey-gssapi-keytab "/usr/local/samba/private/dns.keytab" ;

tkey-domain "nome\_do\_realm\_samba";
	
\};

As variáveis adicionadas no arquivos são para:

\begin{itemize}
	\item{directory} -  É o caminho absoluto do seu servidor dns;
	\item{tkey-gssapi-keytab} - Local da chave do dns para conexão com o kerberos;
	\item{tkey-domain} - Nome do Domínio.
%	\item{auth-nxdomain} - ...
%	\item{listen-on-v6} - ...
\end{itemize}

%\begin{itemize}
%	\item \textbf{\# wget ftp://ftp.isc.org/isc/bind9/9.9.0/bind-9.9.0.tar.gz} - Download da versão 9.9 do bind9;
%	\item \textbf{\# tar xzvf bind-9.9.0.tar.gz} - Descompactar o pacote do bind9;
%	\item \textbf{\# cd bind-9.9.0} - Acessar o diretório do bind9 descompactado;
%	\item \textbf{\# ./configure "--prefix=/usr/local/bind9 "--sysconfdir=/etc/bind} - Configurar os parâmentros para a instalação do bind, tais como o local onde vai ser instalado e onde ficarão os arquivos de configuração;
	% \item \textbf{\# make} - Leitura do comando Makefile;
	% \item \textbf{\# make install} - Executa os comandos configurados para o parâmetro install do arquivo Makefile;
	% \item \textbf{\# cd /etc/bind} - Acessar o diretório onde se encontram os arquivos do bind;
	% \item \textbf{\# vim named.conf} - Cria e edita o arquivo. Adicione as linhas abaixo no arquivo;
	% 	\begin{enumerate}
	% 		\item \textbf{include "/etc/bind/named.conf.options";}
	% 		\item \textbf{include "/etc/bind/named.conf.local";}
	% 	\end{enumerate}
	%  		\item \textbf{\# vim named.conf.options}
	% 		\begin{enumerate}
				% \item \textbf{Adicionar no arquivo} - options \{
%       			
%					directory "/var/cache/bind";
%
%					auth-nxdomain no;
%
%					listen-on-v6 \{ any; \};
%					
%					\};
%			\end{enumerate}
%\end{itemize}

O comando provision gera os arquivos de configuração necessários para o funcionamento do samba com o servidor dns.

\begin{itemize}
	\item \textbf{\# vim named.conf.local} -  Adicione a linha abaixo no arquivo;
		\begin{enumerate}
			\item \textbf{include "/usr/local/samba/private/named.conf";}
		\end{enumerate}
\end{itemize}

Com os arquivos named.conf.local e named.conf.options devidamente criados e configurados, deve-se inclui-los no arquivos named.conf

\# vim named.conf

include "/etc/bind/named.conf.local";
include "/etc/bind/named.conf.options";


Como o samba 4 já vem com configurações prontas do bind9 é necessário escolher qual a versão do dns que esta sendo utilizada.

\begin{itemize}
	\item \textbf{\# vim /usr/local/samba/private/named.conf}
\end{itemize}

*****FIGURA DO ARQUIVO******

\# For BIND 9.8.0

    \# database "dlopen /usr/local/samba/lib/bind9/dlz\_bind9.so"; 

Descomentar

 \# For BIND 9.9.0

    database "dlopen /usr/local/samba/lib/bind9/dlz\_bind9\_9.so";

\begin{itemize}
	\item \textbf{\# groupadd named \&\& useradd named -g named} - Cria o usuário responsável pelo bind e o insere no grupo named;
	\item{\# chown named:named /usr/local/samba/private/dns.keytab}
	\item \textbf{\# /usr/local/bind9/sbin/named -u named -g} - Inicia o bind com o usuário named;
\end{itemize}

O servidor samba tem que ter seu endereço DNS configurado para apontar para seu servidor DNS.

\begin{itemize}
	\item \textbf{\# echo 'nameserver ip\_do\_servidor' $>$$>$ /etc/resolv.conf} - Define o endereço do servidor de DNS que o computador irá enviar suas solicitações;
\end{itemize}

A partir de agora para acessar a internet através do servidor samba o bind deverá estar sendo executado.

\section{Instalação do Kerberos}

Segundo(http://samba4.wordpress.com/2009/09/25/instalacao-samba4) Autenticação Kerberos é um protocolo de rede. Foi concebido para fornecer autenticação forte para o cliente/servidores de aplicativos usando criptografia de chaves secretas, então um cliente pode provar a sua identidade para um servidor (e vice-versa) em uma conexão de rede insegura.
Em nosso caso utilizaremos BIND com suporte ao Heimdal Kerberos por causa do GSS-TSIG algoritmo de serviço de segurança genérico para autenticação de transação com chave secreta de DNS (GSS-TSIG) este mecanismo é utilizado para estabelecer relações TSIG para autenticação do tipo Kerberos, necessário para interagir BIND com Samba4, com essas credenciais o DNS aceita atualizações GSS-TSIG assinadas e verifica as credenciais de correspondentes com as credencias cadastradas no Samba4, isso permite aos usuários descarregar o DNS dos usuários do Microsoft Windows sem ter a segurança comprometida.

\begin{itemize}
	\item \textbf{\# apt-get install krb5-user krb5-kdc krb5-config kstart} - Instala todos os pacotes necessários e faz as referências necessárias.
\end{itemize}

Após instalar os pacotes, substitua o /etc/krb5.conf pelo arquivo criado e pré-configurado pelo samba que esta localizado em /usr/local/samba/private/krb5.conf

\begin{itemize}
	\item \textbf{\# cp /usr/local/samba/private/krb5.conf  /etc/}
\end{itemize}

Teste para verificar se todos as configurações foram realizadas corretamente

\begin{itemize}
	\item \textbf{\# host -t SRV \_ldap.\_tcp."nome do realm sem aspas".} - O resultado deve ser parecido : \textbf{\_ldap.\_tcp."nome do realm sem aspas" has SRV record 0 100 389 server."nome do realm sem aspas".}
	\item \textbf{\# host -t SRV \_kerberos.\_udp."nome do realm sem aspas".} - O resultado deve ser parecido : \textbf{\_kerberos. \_udp."nome do realm sem aspas" has SRV record 0 100 88 server."nome do realm sem aspas".}
	\item \textbf{\# host -t A "nome do realm sem aspas"} - O resultado deve ser parecido : \textbf{"nome do realm sem aspas" has address "ip do servidor}
\end{itemize}

\section{Kerberos com Bind9}

Configurar atualizações dinâmicas no DNS com o kerberos

Para o funcionamento das atualizações algumas variáveis necessárias de sistema devem ser criadas para o acesso do kerberos com bind

\begin{itemize}
	\item \textbf{\# KEYTAB\_FILE="/usr/local/samba/private/dns.keytab"}
	\item \textbf{\# KRB5\_KTNAME="/usr/local/samba/private/dns.keytab"}
	\item \textbf{\# export KEYTAB\_FILE}
	\item \textbf{\# export KRB5\_KTNAME}
\end{itemize}

Mudar o dono e o grupo do dns.keytab para que o bind possa alterar o arquivo

\begin{itemize}
	\item \textbf{\# chown named:named /usr/local/samba/private/dns.keytab}
	\item \textbf{\# /usr/local/samba/sbin/samba\_dnsupdate "--verbose} - Atualização automática do dns do samba.
\end{itemize}

\section{AD}

Pacotes necessários para gerenciar o AD no windows XP ou windows server.

\section{GPO}

Pacotes necessários para gerenciar as GPO's no windows XP ou windows server.

\section{Compartilhamento de arquivos e impressoras}

SAMBA4 ainda não consegue compartilhar arquivos e impressoras de forma fácil e simplificada como o samba 3, e tem problemas com a integração dos usuários e grupos do Active Directory com os locais, dificultando a definição das permissões a arquivos e diretórios.

Uma solução para tal problema é identificar o código do usuário no Active Directory e dar as devidas permissões a pasta desejada.

\begin{itemize}
	\item \textbf{\# /usr/local/samba/bin/wbinfo "--name-to-sid USERNAME} - O resultado deve ser o sid do usuário no samba. Exemplo : S-1-5-21-4036476082-4153129556-3089177936-1005 SID\_USER(1)
	\item \textbf{\# /usr/local/samba/bin/wbinfo "--sid-to-uid S-1-5-21-4036476082-4153129556-3089177936-1005} - Mostra o id do usuário e é a referência do usuário local com o do samba 4.
	\item \textbf{\# chown 3000011 /pasta\_que\_será\_compartilhada} - Mudando o usuário do diretório e as suas permissões, o usuário do AD irá ter o acesso aos arquivos.
\end{itemize} 

\section{Gerenciando o Samba4}

O samba-tools - Gerência o samba. Com ele se poder criar usuários, grupos, gpo's e outras funções do Active Directory, porém um forma de texto.

***FIGURA DO SAMBA-TOOLS***

\section{Maquinas linux e samba3 interagindo com o Active Directory do  Samba4}

Segundo \cite{UBUNTU-WIKI} \ref{UBUNTU-WIKI} a forma de incluir uma maquina Ubuntu no Active Directory é modificar alguns arquivos de configuração.
Segue abaixo os arquivos e os procedimentos.

\textbf{Informações}

\begin{itemize}
	\item \textbf{fja.br} -  Domínio do Active Directory
	\item \textbf{fjadc01.fja.br} - Controlador de domínio
	\item \textbf{10.1.0.1} - IP do controlador de domínio
	\item \textbf{FJA.BR} - Kerberos Realm
	\item \textbf{gert} - Estação de Trabalho Ubuntu
	\item \textbf{gert.fja.br} - FQDN da estação de trabalho
	\item \textbf{fjadc01} - Servidor NTP
\end{itemize}

\textbf{Instalando os pacotes necessários}

\begin{itemize}
	\item {\# aptitude install krb5-user libpam-krb5 winbind samba smbfs smbclient krb5-config libkrb53 libkadm55 vim}
\end{itemize}

\textbf{Sincronizando a hora}

\begin{itemize}
	\item {\# ntpdate 10.2.0.1}
\end{itemize}

\textbf{Edite o arquivo /etc/hosts adicionando o ip e o nome do DC de sua rede}

\begin{itemize}
	\item {\# vim /etc/hosts}
\end{itemize}

127.0.0.1       gert.fja.br localhost gert

127.0.1.1       gert

\# The following lines are desirable for IPv6 capable hosts

::1     ip6-localhost ip6-loopback

fe00::0 ip6-localnet

ff00::0 ip6-mcastprefix

ff02::1 ip6-allnodes

ff02::2 ip6-allrouters

ff02::3 ip6-allhosts

10.2.0.1   fjadc01

10.2.0.2   fjadc02

\textbf{Configurando o Kerberos}

\begin{itemize}
	\item {\# vim /etc/krb5.conf}
\end{itemize}

[libdefaults]

	default\_realm = FJA.BR

[realms]

    FJA.BR = \{

      kdc = fjadc01.fja.br

      default\_domain = FJA.BR

      kpasswd\_server = fjadc01.fja.br

      admin\_server = fjadc01.fja.br

     \}

[domain\_realm]

.fja.br = FJA.BR

\textbf {Testando a conexão com o Active Directory}

\begin{itemize}
	\item {kinit $<$ENTER$>$}
	\item {Password for alex$@$FJA.BR: ****}
	\item {klist $<$ENTER$>$}
	\item {Ticket cache: FILE:/tmp/krb5cc\_1000}
	\item {Default principal: alex$@$FJA.BR}
\end{itemize}

\textbf {Se o resultado for este o Kerberos está funcionando corretamente}

	Valid starting Expires Service principal 07/16/07 15:48:35  07/17/07 01:49:08  

	krbtgt/FJA.BR@FJA.BR renew until 07/17/07 15:48:35
	
	Kerberos 4 ticket cache: /tmp/tkt1000
	
	klist: You have no tickets cached

\textbf{Acessando o Domínio}

\begin{itemize}
	\item {\# vim /etc/samba/smb.conf} -  Adicione as seguintes linhas
\end{itemize}

[global]

        security = ads

        realm = FJA.BR

        password server = 10.2.0.1

        workgroup = ADMINISTRATIVO

\#       winbind separator = +

        idmap uid = 10000-20000

        idmap gid = 10000-20000

        winbind enum users = yes

        winbind enum groups = yes

        template homedir = /home/\%D/\%U

        template shell = /bin/bash

        client use spnego = yes

        client ntlmv2 auth = yes

        encrypt passwords = yes

        winbind use default domain = yes

        restrict anonymous = 2

\# to avoid the workstation from

\# trying to become a master browser

\# on your windows network add the

\# following lines

        domain master = no

        local master = no

        preferred master = no

        os level = 0

\textbf{Reinicie os serviços}

\begin{itemize}
	\item \textbf{\# /etc/init.d/winbind stop}
	\item \textbf{\# /etc/init.d/samba restart}
	\item \textbf{\# /etc/init.d/winbind start}
\end{itemize}

\textbf{Adicione a conta ao domínio}

\begin{itemize}
	\item \textbf{\# net ads join}
	\item \textbf{Resultado} - Using short domain name – GERT Joined 'GERT' to realm 'FJA.BR'
\end{itemize}

\textbf{Configure a Autenticação}

\begin{itemize}
	\item \textbf{\# vim /etc/nsswitch.conf}
\end{itemize}

	passwd:         compat winbind

	group:          compat winbind

	shadow:         compat

\textbf{Teste o winbind}

\begin{itemize}
	\item {getent passwd}
\end{itemize}

quiosque:*:10018:10000:Quiosque:/home/ADMINISTRATIVO/quiosque:/bin/bash

\begin{itemize}
	\item {getent group}
\end{itemize}

\_\_coordenação de enfermagem:x:10046:coordenf

\_\_coordenação de design:x:10047:smarino,coorddes

\textbf{Configure o PAM}

\begin{itemize}
	\item {\# vi /etc/pam.d/common-account} - Adicione as seguintes linhas
\end{itemize}

account sufficient       pam\_winbind.so

account required         pam\_unix.so

\begin{itemize}
	\item {\# vim /etc/pam.d/common-auth} - Adicione as seguintes linhas
\end{itemize}

auth sufficient pam\_winbind.so

auth sufficient pam\_unix.so nullok\_secure use\_first\_pass

auth required   pam\_deny.so

\begin{itemize}
	\item {\# vim /etc/pam.d/common-session} Adicione as seguintes linhas
\end{itemize}

session required pam\_unix.so

session required pam\_mkhomedir.so umask=0022 skel=/etc/skel

\begin{itemize}
	\item {/etc/pam.d/sudo} - Adicione as seguintes linhas
\end{itemize}

auth sufficient pam\_winbind.so

auth sufficient pam\_unix.so use\_first\_pass

auth required   pam\_deny.so

$@$include common-account

%Criando o HOMEDIR do dominio

%sudo mkdir /home/ADMINISTRATIVO

\textbf{Reinicie os serviços}

\begin{itemize}
	\item \textbf{\# /etc/init.d/winbind stop}
	\item \textbf{\# /etc/init.d/samba restart}
	\item \textbf{\# /etc/init.d/winbind start}
\end{itemize}

\textbf{Logando no domínio}

Vá para a console usando o comando CTRL+ALT+F1 e logue no sistema com o login e senha do dominio

\begin{itemize}
	\item {login: nome\_do\_usuário}
	\item {Password: *****}
	\item {nome\_do\_usuário$@$gert:~\$}
\end{itemize}

\section{Script para adicionar maquina linux no Active Directory}

\#!/bin/sh
 
\#\#\#\#\#\#\#\#\#\#\#\#\#\#\#\#\#\#\#\#\#\#\#\#\#\#\#\#\#\#\#\#\#\#\#\#\#\#\#\#\#\#\#\#\#\#\#\#\#\#\#\#\#\#\#\#\#\#\#\#\#\#\#\#\#\#

\# Copyright (C) 2011 - Fabio Antonio Ferreira \hspace{160pt} \#

\# http://fantonio.wordpress.com | fantonios@gmail.com \hspace{106pt} \#

\# Este trabalho está licenciado sob uma Licença Creative Commons \hspace{59pt} \#

\# Atribuição-Compartilhamento pela mesma Licença 2.5 Brasil. Para ver a copia \#

\# desta licença, acesse: http://creativecommons.org/licenses/by-sa/2.5/br/ \hspace{34pt} \#

\# ou envie uma carta para Creative Commons, 171 Second Street, Suite 300, \hspace{19pt} \#

\# San Francisco, California 94105, USA. \hspace{188pt} \#

\# Modificações em 27 de Julho de 2012 por Gabriel Rocha (GBR) \hspace{67pt}             \#

\# email: gabriel.rocha.gbr@gmail.com \hspace{198pt} \#

\#\#\#\#\#\#\#\#\#\#\#\#\#\#\#\#\#\#\#\#\#\#\#\#\#\#\#\#\#\#\#\#\#\#\#\#\#\#\#\#\#\#\#\#\#\#\#\#\#\#\#\#\#\#\#\#\#\#\#\#\#\#\#\#\#\#

\# == FUNCOES ===============================================

USUARIO=`whoami`

if [ "\$USUARIO" != "root" ]; then

 	echo

  echo "======================================================"

  echo " ESTE PROGRAMA PRECISA SER EXECUTADO COM PERMISSOES DE SUPERUSUARIO!"  

  echo " Abortando..."

  echo "======================================================"

  echo

  exit 1

fi

\_HEAD () \{

`which clear`

echo "========================================================="

echo "SISTEMA PARA ADICIONAR MAQUINA LINUX AO DOMÍNIO WINDOWS OU LINUX"

echo "========================================================="

\}

\_PACOTES () \{

        echo "Instalando os pacotes necessários";       


  apt-get install krb5-user libpam-krb5 winbind samba smbfs smbclient krb5-config libkrb53 libkdb5-4 libgssrpc4 -y $>$ /dev/null;
  
      check=\$(echo \$?)

        if [ \$check -eq 0 ]; then

           echo "Pacotes instalados com sucesso"

        else

           echo "Falha ao instalar os pacotes"

        fi

\}

\_HORA () \{

        echo "Atualizando data e hora";

        ntpdate br.pool.ntp.org $>$ /dev/null;

        echo "Horario atual:" `date`

        echo "Hora alterada com sucesso"

\}

\_BACKUP\_ORIG () \{

  \# Rotina de Backup dos arquivos de configurações.

	if [ ! -e /etc/krb5.conf\_backup ]; then

		cp /etc/krb5.conf /etc/krb5.conf\_backup $>$ /dev/null;

	fi

	if [ ! -e /etc/resolv.conf\_backup ]; then

		cp /etc/resolv.conf /etc/resolv.conf\_backup $>$ /dev/null

	fi

	if [ ! -e /etc/samba/smb.conf\_backup ]; then

        	cp /etc/samba/smb.conf /etc/samba/smb.conf\_backup $>$ /dev/null

	fi

	if [ ! -e /etc/nsswitch.conf\_backup ]; then

        	cp /etc/nsswitch.conf /etc/nsswitch.conf\_backup $>$ /dev/null

	fi

	if [ ! -e /etc/pam.d/common-account\_backup ]; then

	        cp /etc/pam.d/common-account /etc/pam.d/common-account\_backup $>$ /dev/null

	fi

	if [ ! -e /etc/pam.d/common-auth\_backup ]; then

	        cp /etc/pam.d/common-auth /etc/pam.d/common-auth\_backup $>$ /dev/null

	fi

	if [ ! -e /etc/pam.d/common-session\_backup ]; then

	        cp /etc/pam.d/common-session /etc/pam.d/common-session\_backup $>$ /dev/null

	fi

	if [ ! -e /etc/pam.d/sudo\_backup ]; then

	        cp /etc/pam.d/sudo /etc/pam.d/sudo\_backup $>$ /dev/null

	fi
         
        check=\$(echo \$?)

   if [ \$check -eq 0 ]; then

      echo "Rotina de Backup executada com sucesso!"

   else

      echo "Falha ao fazer o Backup."

   fi
         
\}

\_RETURN\_BACKUP () \{

        \# Rotina de Recuperação do Backup de configurações.

        mv /etc/krb5.conf\_backup /etc/krb5.conf $>$ /dev/null

        mv /etc/resolv.conf\_backup /etc/resolv.conf $>$ /dev/null

        mv /etc/samba/smb.conf\_backup /etc/samba/smb.conf $>$ /dev/null

        mv /etc/nsswitch.conf\_backup /etc/nsswitch.conf $>$ /dev/null

        mv /etc/pam.d/common-account\_backup /etc/pam.d/common-account $>$ /dev/null

        mv /etc/pam.d/common-auth\_backup /etc/pam.d/common-auth $>$ /dev/null
        
        mv /etc/pam.d/common-session\_backup /etc/pam.d/common-session $>$ /dev/null

        mv /etc/pam.d/sudo\_backup /etc/pam.d/sudo $>$ /dev/null
         
        
				check=\$(echo \$?)

   if [ \$check -eq 0 ]; then

      echo "Recuperação do Backup executada com sucesso!"

   else

      echo "Falha na recuperação do Backup."

   fi
         
\}

\_NOME\_DOMINIO () \{
 
   \#Entrada do nome do dominio ao qual deseja engreçar.

	 \#No caso do linux temos dois servidores um do KDC e outro do dominio

	 \#No windows informamos o servidor kdc

    read -p "Entre com o nome do Domínio:" var1

    dominio=\$(echo \$var1 | tr a-z A-Z)

    read -p "Entre com o seu KDC (key Distribution Center):" var2

    kdc=\$(echo \$var2 | tr A-Z a-z)         

\}

\_IP\_DNS (){

	\#IP do servidor de dns

	read -p "Entre com o IP do servidor de DNS:" ip

	echo "nameserver \$ip" $>$ /etc/resolv.conf

\}

\_SO\_SERVIDOR () \{

	\#Sistema Operacional do AD	

	read -p "Entre com o S.O. do servidor (Linux ou Windows): " so

	so=\$(echo \$so | tr a-z A-Z)

	workgroup=""

	if [ \$so = "LINUX" ] ; then

		read -p "Informe o Domain do Samba4: " workgroup

		workgroup=\$(echo \$workgroup | tr a-z A-Z)

	else

		workgroup=\$(echo \$var1)

	fi

\}

\_KRB5 () \{

   echo "[libdefaults]

   default\_realm = \$dominio

	 \# The following krb5.conf variables are only for MIT Kerberos.

      krb4\_config = /etc/krb.conf

      krb4\_realms = /etc/krb.realms

      kdc\_timesync = 1

      ccache\_type = 4

      forwardable = true

      proxiable = true

		\# The following libdefaults parameters are only for Heimdal Kerberos.

      v4\_instance\_resolve = false

      v4\_name\_convert = \{

           host = \{

               rcmd = host

               ftp = ftp

           \}  

           plain = \{

               something = something-else

           \}  

      \}  

      fcc-mit-ticketflags = true

   [realms]

   		\$dominio = \{

        	kdc = \$kdc
           
            admin\_server = \$kdc

           \}  
             
   [domain\_realm]

   		.\$var1 = \$kdc

   [login]

   		krb4\_convert = true

   		krb4\_get\_tickets = false" $>$ /etc/krb5.conf
 
   echo "Configuração alterada com sucesso!"

\}

\_TESTEAD () \{

   read -p "Entre com um usuário para testar sua conexão com o Active Directory:" user

   kinit \$user$@$\$dominio

    

   check=\$(echo \$?)

   if [ \$check -eq 0 ]; then

      echo "Sua máquina conectou com sucesso!"

   else

      echo "Falha ao se conectar com o Active Directory"

   fi

\}

\_SMB () \{

    

   maquina=\$(hostname)

   echo "\# Sample configuration file for the Samba suite for Debian GNU/Linux.

   \#======================= Global Settings =======================

   [global]

      workgroup = \$workgroup

      netbios name = \$maquina

      realm = \$var1

      server string = \% h Server

      dns proxy = no

  	  log file = /var/log/samba/log.\%m  

	  max log size = 1000

	  syslog = 0  

      panic action = /usr/share/samba/panic-action \%d

      security = ADS

      password server = \$kdc

      encrypt passwords = true

      passdb backend = tdbsam

      obey pam restrictions = yes

      unix password sync = yes

      passwd program = /usr/bin/passwd \%u
      
      pam password change = yes

      idmap uid = 10000-20000

      winbind gid = 10000-20000

      winbind enum users = yes

      winbind enum groups = yes

      winbind use default domain = yes

      template homedir = /home/\%D/\%U

      template shell = /bin/bash

   [homes]

      comment = Home Directories

      browseable = no

      read only = yes

      create mask = 0700

      directory mask = 0700

      valid users = \%S " $>$ /etc/samba/smb.conf

   echo "Configuração alterada com sucesso!"

\}

\_FUNC\_RESTART() \{

        \# Stop Winbind

        /etc/init.d/winbind stop $>$ /dev/null

        check=\$(echo \$?)

   if [ \$check -eq 0 ]; then

      echo "Winbind Stop!"

   else

      echo "Falha ao parar o Winbind"

   fi

     \# Restart Samba

     /etc/init.d/smbd restart $>$ /dev/null

     check=\$(echo \$?)

   if [ \$check -eq 0 ]; then

      echo "Samba restart com sucesso!"

   else

      echo "Falha no restart do Samba!"

   fi

    \# Start Winbind

    /etc/init.d/winbind start $>$ /dev/null

    check=\$(echo \$?)

   if [ \$check -eq 0 ]; then

      echo "Winbind start!"

   else

      echo "Falha ao fazer iniciar o Winbind!"

   fi

\}

\_ADDDOMINIO () \{
    
 
  echo "++++++++++++++++++++++++++++++++++++++++++++"

   echo "++  Adicionando a Máquina no Domínio  ++"

   echo "++++++++++++++++++++++++++++++++++++++++++++"

   \# Adicionando a máquina ao domínio

        read -p "Entre com um usuário administrador de Domínio:" user   

   net ads join -U \$user;

        check=\$(echo \$?)

        clear

        \# Validação da conexão com o domínio

        if [ \$check -eq 0 ]; then

      echo "Sua máquina foi adicionada no Domínio!"

   else

      echo "Falha ao adicionar a máquina no Domínio"

   fi

\}

\_TESTDOMINIO () \{

        \# Teste de requisição ao dominio

        wbinfo -t $>$ /dev/null

        check=\$(echo \$?)

   if [ \$check -eq 0 ]; then

      echo "Teste de Domínio!"

   else

      echo "Falha ao testar o Domínio"

   fi

\}

\_FUNCAUTENTICACAO () \{

        \# Configurando o arquivo nsswitch.conf

        echo "passwd:         compat winbind

              group:          compat winbind

              shadow:         compat" $>$ /etc/nsswitch.conf

        \# Teste de configuração do Winbind        

        check=\$(echo \$?)
   		
		if [ \$check -eq 0 ]; then

      echo "Winbind testado com sucesso!"

   else

      echo "Falha ao testar o Winbind"

   fi

        \# PAM - common-account

        echo "account sufficient       pam\_winbind.so
              account required         pam\_unix.so" $>$ /etc/pam.d/common-account

        \# PAM - common-auth

        echo "auth sufficient pam\_winbind.so

              auth sufficient pam\_unix.so nullok\_secure use\_first\_pass

              auth required   pam\_deny.so" $>$ /etc/pam.d/common-auth

        \# PAM - common-session      

        echo "session required pam\_unix.so

              session required pam\_mkhomedir.so umask=0022 skel=/etc/skel" $>$ /etc/pam.d/common-session

        \# PAM - sudo

        echo "auth sufficient pam\_winbind.so

              auth sufficient pam\_unix.so use\_first\_pass

              auth required   pam\_deny.so

              $@$include common-account" $>$ /etc/pam.d/sudo

        \# Teste de configuração do PAM

        check=\$(echo \$?)

   if [ \$check -eq 0 ]; then

      echo "PAM configurado com sucesso!"

   else

      echo "Falha ao configurar o PAM"

   fi

\}

\_FUNC\_HOMEDIR () \{

        HOME\_DIR=\$var1

        if [ -d /home/\$HOME\_DIR ]; then

                echo "Já existe este diretório !"                

        else

                echo "Este diretório não existe !"

                echo "Criando o diretório \$HOME\_DIR"

      mkdir /home/\$var1

                sleep 2

        fi

\}

\_FUNC\_DEL\_MAQ\_DOMINIO () \{

    

   maquina=\$(hostname)

        echo "++++++++++++++++++++++++++++++++++++++++++++"

        echo "++  Removendo a Máquina no Domínio  ++"

        echo "++++++++++++++++++++++++++++++++++++++++++++"
       
        \# Remover a máquina ao domínio

        read -p "Entre com um usuário administrador de Domínio:" user

   net ads status -U \$user

   check1=\$(echo \$?)   

   clear

   \# Validação se a máquina está no domínio

   if [ \$check1 -eq 255 ]; then

      echo "A máquina \$maquina não está no dominio"

   else

      \# Validação de remoção de máquina do domínio

      net ads leave -U \$user;

      check=\$(echo \$?)

      clear

      if [ \$check -eq 0 ]; then

         echo "Sua máquina foi removida do Domínio!"

      else

         echo "Falha ao remover a máquina no Domínio"

      fi

   fi

\}

\# =========================================================

\# Menu de seleção

echo "Linux Active Directory:"

echo "(1) Adicionar Máquina no Domínio"

echo "(2) Remover Máquina do Domínio"

echo "(3) Verificar conexão com o Domínio"

echo "(0) Sair"

echo "Digite a opção desejada:"

read resposta

case "\$resposta" in

        1)  

      \_HEAD

      \_PACOTES

      \_HORA

      \_BACKUP\_ORIG

      \_NOME\_DOMINIO

      \_IP\_DNS

      \_SO\_SERVIDOR

      \_KRB5

      \_TESTEAD

      \_SMB

      \_FUNC\_RESTART

      \_ADDDOMINIO

      \_TESTDOMINIO

      \_FUNCAUTENTICACAO

      \_FUNC\_RESTART

      echo "++++++++++++++++++++++++++++++++++++++++++++"

      echo "++ Bem vindo ao dominio \$dominio ++"

      echo "++++++++++++++++++++++++++++++++++++++++++++"

                ;;  

        2)  

       \_FUNC\_DEL\_MAQ\_DOMINIO

		\_RETURN\_BACKUP

                ;;  

        3)  

       \_TESTDOMINIO

                ;;  

        0)  

                exit

                ;;  

        \*)  

                echo 'Opção Inválida!'

esac

\section{Windows no domínio Samba 4}