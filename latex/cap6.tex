\chapter{CONCLUSÕES}

Neste trabalho foi apresentada uma proposta de implantação de um servidor de modo a melhorar o acesso dos usuários da instituto aos recursos de autenticação dos usuários, compartilhamento de arquivos e compartilhamento de impressoras na rede e ainda assegurar a disponibilidade dos mesmos, independente do equipamento utilizado. Antes da implantação dos servidores todos os serviços eram atrelados à máquina local do usuário assim impedindo que o usuário a usasse em outra máquinas da rede.

Além disso, foi possível mostrar que, ao realizar esta implantação, o administrador de rede terá um maior controle dos acessos dos usuários já que a conta do usuário não fica mais atrelada uma única máquina e o Samba 4 oferece uma opção de visualizar todos os usuários cadastrados e podendo permitir ou negar recursos, por exemplo.

Foi possível ainda discutir sobre ferramentas disponíveis para a realização da implantação da proposta, de forma simples e objetiva, focada na estrutura abordada para receber o servidor, além de configurações e \textit{scripts} criados para auxiliar o processo.

%O Samba 3 é mais estável, e portanto é mais recomendado em redes de médio porte, porém não se comporta como \textit{Active Directory} e não permite \textit{polices}, mas realiza a autenticação dos usuários, compartilhamento de arquivos e impressoras.
%O Samba 3 e Samba 4 não podem ser instalados e configurados no mesmo servidor por trabalharem com os mesmos daemons de inicialização, um serviço quando iniciado anula o outro.	
%Por ainda terem funções distintas a melhor forma de se trabalhar com os dois na mesma rede é trabalhando em servidores distintos porém interligados.

O Samba 4 funcionou perfeitamente no que se propõe a fazer, mesmo sendo uma versão \textit{Release Candidate} e com o lançamento de uma versão estável o Samba 4 irá tornar o Samba 3 desnecessário.

Como em qualquer trabalho que envolve ferramentas em evolução, neste trabalho serão necessárias melhorias e novas pesquisas, não permitindo que o mesmo fique ultrapassado e não seja compatível com as ferramentas em constante atualização.




