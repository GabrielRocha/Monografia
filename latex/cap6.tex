\chapter{CONCLUSÕES}

\section{Objetivos alcançados}

Neste trabalho foi apresentada uma proposta de implementação de um servidor de modo a otimizar o acesso dos usuários da instituição de ensino aos recursos disponíveis na rede e ainda assegurar a disponibilidade destes recursos, independente do equipamento utilizado.

Além disso, foi possível mostrar que, ao realizar esta implementação, o administrador de rede terá um maior controle dos acessos dos usuários, podendo permitir ou negar recursos, por exemplo.

Foi possível ainda apresentar ferramentas disponíveis para a realização da implementação da proposta, de forma simples e objetiva, focada na estrutura abordada para receber o servidor, além de configurações e scripts criados para otimizar o processo.

\section{Trabalhos futuros}

Como em qualquer trabalho que envolve ferramentas em evolução, neste trabalho serão necessárias melhorias e novas pesquisas, não permitindo que o mesmo fique ultrapassado e não seja compatível com as ferramentas em constante atualização.

Neste trabalho foi utilizada uma versão ainda em desenvolvimento do Samba 4, portanto uma proposta de trabalho futuro seria realizar maiores testes em ambiente de produção com a versão estável do mesmo.
