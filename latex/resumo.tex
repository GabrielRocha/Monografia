\begin{center}
\textbf{RESUMO}
\end{center}
\singlespacing

Este trabalho sugere uma proposta de implementação de um servidor de compartilhamento de arquivos, impressoras e um Active Directory em uma instituição de ensino com a missão de facilitar o compartilhamento dos recursos de rede disponíveis e tornar mais seguro e confiável o controle de acesso dos usuários a estes recursos. Também é possível encontrar conceitos básicos para a compreensão das ferramentas utilizadas além de passo-a-passo e scripts necessários para realizar a implementação de toda a estrutura na rede.

\noindent PALAVRAS-CHAVE: Linux, Samba, PDC, Compartilhamento, LDAP, Active Directory
%Com o avanço da tecnologia, é cada vez mais comum a disponibilização de conteúdo acadêmico (artigos, trabalhos de conclusão de curso, teses, etc.) na internet para fins de pesquisa. Grande parte desses documentos é encontrada em formato PDF (Portable Document Format), por ser um formato largamente conhecido e muito utilizado para leitura de arquivos em tela, preservando sua qualidade e formatação independentemente do computador ou do sistema operacional utilizado.
 % Tendo em vista esse grande volume de documentos disponíveis, é comum que não haja uma preocupação por parte de seus autores em nomeá-los de acordo com o título do conteúdo ou com o assunto abordado. Baixar um arquivo com nome incompatível com seu conteúdo não seria problema se esse arquivo não se misturasse a vários outros em mesma situação no disco de armazenamento do usuário, o que fatalmente acontece.