\begin{center}
\textbf{RESUMO}
\end{center}
\singlespacing

\noindent 
	O objetivo deste trabalho foi desenvolver uma ferramenta que possa identificar os metadados de um arquivo PDF com formatação de artigo, respeitando da norma ABNT, (título, palavras-chave, autor, etc.) e utilizá-los para renomear o arquivo de modo a facilitar seu reconhecimento, visando uma maior organização e distinção do mesmo.
 \\
\noindent PALAVRAS-CHAVE: Metadados, PDF, Extração de dados, Desenvolvimento de Software
%Com o avanço da tecnologia, é cada vez mais comum a disponibilização de conteúdo acadêmico (artigos, trabalhos de conclusão de curso, teses, etc.) na internet para fins de pesquisa. Grande parte desses documentos é encontrada em formato PDF (Portable Document Format), por ser um formato largamente conhecido e muito utilizado para leitura de arquivos em tela, preservando sua qualidade e formatação independentemente do computador ou do sistema operacional utilizado.
 % Tendo em vista esse grande volume de documentos disponíveis, é comum que não haja uma preocupação por parte de seus autores em nomeá-los de acordo com o título do conteúdo ou com o assunto abordado. Baixar um arquivo com nome incompatível com seu conteúdo não seria problema se esse arquivo não se misturasse a vários outros em mesma situação no disco de armazenamento do usuário, o que fatalmente acontece.